\documentclass[12pt]{amsart}
\usepackage{fullpage}
 
\title{CS 6505 - Homework 2}
\author{Ken Brooks}
\date{\today}
 
\begin{document}
\maketitle

\begin{itemize}
\item[{\rm 5.}] Another alternative Turing machine model has a single, one-way infinite tape, but two read-write heads.  The transition function has the form $\delta:Q\times\Gamma^{2}\rightarrow Q\times\Gamma^{2}\times\{L,R,S\}^{2}$, the same as a multi-tape machine.  Describe how you would program such a machine to decide the language $L=\{ww \mid w \in\{0,1\}^{*}\}$. 
\\[.2in]Hint: You can use the S (stay put) movement to achieve the effect of having one head move faster than the other. 
\\[.2in](Writing down the transition function is not required, but you should go into enough detail to convince the reader that you could do so if asked.)
\\[.2in]\textbf{Answer:}
\vspace{.2in}
\begin{itemize}
\item \textbf{State 1}: Start state. 
\begin{itemize}
\item If blank, accept
\end{itemize}
\item \textbf{State 2}: Shift string right. 
\begin{itemize}
\item Insert special character at beginning "\#"
\item Shift string to the right
\end{itemize}
\item \textbf{State 3:} Find middle.
\begin{itemize}
\item Move 1::2 to the end, when H2 sees a blank
\end{itemize}
\item \textbf{State 4}: Check if odd.
\begin{itemize}
\item  H2L, if b, it's even and move back prior to rewind
\item  H2L, if it's not blank, it's odd, and fail
\end{itemize}
\item \textbf{State 5}: Rewind.
\begin{itemize}
\item Backtrack and step H1 fwd 1
\end{itemize}
\item \textbf{State 6}: Match characters.
\begin{itemize}
\item Compare H1 and H2, if match good, if not, fail, move both right 1
\end{itemize}
\end{itemize}
\vspace{.2in}
\item[{\rm 6.}] Suppose we have a one-tape Turing machine M whose head instead of having to move just left or right in each computation step, can move left or right or stay put.  We called these “stay-put machine” in the lesson.  Prove that it is possible to create a new Turing machine $M'$ that recognizes the same language as $M$ by changing only the transition function $\delta$, keeping the same Q and Γ.
\\[.2in]\textbf{Answer:}
\vspace{.2in}
\\For every place a "Stay" move is desired in the transition function, insert two elements
\begin{itemize}
\item A right move reading the element under the head and writing that same element.
\item A left move that also reads the element under the head and writes that same element
\end{itemize}
\vspace{.2in}
\item[{\rm 7.}] Suppose that we constrained the standard 1-tape Turing machine to only be able to write to each square on the tape two times.  Prove that this model can decide every language that a standard Turing machine can.
\\[.2in]Hint: Use lots of tape and do a lot of copying.
\\[.2in]\textbf{Answer:}
\\[.2in] In concept we process the tape, writing to a clear space off the right of the original string and marking off characters from the original string as we go. This involves:
\begin{itemize}
	\item Writing a separator to the right of the original string.
	\item Copying the tape up to the head position onto the new part of the tape, and marking off the symbols copied.
	\item Writing the symbol required by the TM in the new position, marking the head position, and marking off the original symbol.
	\item Copying the rest of the the tape over, and again marking off the symbols copied
	\item Resuming the computations at the original state and marked tape position on the freshly copied tape.
\end{itemize}
\vspace{.2in}
Bonus (hard): Describe how you would modify your machine so that it only writes to each square once.
\\[.2in] \textbf{Answer:}
\\[.2in]The single write problem...
\vspace{.2in}
\item[{\rm 8.}] Consider a Turing machine with a two-dimensional tape where the head can move up and down as well as right and left.  Assume that the paper is infinite in the right and downward directions.  Give a formal definition of this machine
\\[.2in]\textbf{Answer:}\footnote{Definition modfied from Sipser, Michael, \textit{Introduction to the Theory of Computation, 3e}, p. 168}
\vspace{.2in}
\\ A Turing machine is a 7-tuple, $(Q, \Sigma, \Gamma, \delta, q_{0}, q_{accept}, q_{reject})$, where Q, $\Sigma$, $\Gamma$ are all finite sets and
\begin{enumerate}
\item Q is the set of states,
\item $\Sigma$ is the input alphabet not containing the blank symbol $\sqcup$, 
\item $\Gamma$ is the tape alphabet, where $\sqcup \in \Gamma$ and $\Sigma \subseteq \Gamma$,
\item $\delta:Q\times\Gamma \rightarrow Q\times\Gamma \times\{L,R,U,D\}$ is the transition function, where allowed moves are L-left, R-right, U-up, D-down.
\item $q_{0} \in Q$ is the start state,
\item $q_{accept} \in Q$ is the accept state, and
\item $q_{reject} \in Q$ is the reject state, where $q_{reject} \neq q_{accept}$.
\end{enumerate}

\vspace{.2in}

\end{itemize}
 
 
\end{document}