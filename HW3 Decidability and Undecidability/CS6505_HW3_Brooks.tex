\documentclass[12pt]{article}
\usepackage{amsmath,amssymb,amsthm}
\usepackage{graphicx}
\usepackage{mathtools}
\usepackage[margin=1in]{geometry}
\usepackage{fancyhdr}
\usepackage{hyperref}
\usepackage{listings}
% \usepackage{endnotes}
\lstloadlanguages{Python}
\setlength{\parindent}{0pt}
\setlength{\parskip}{5pt plus 1pt}
\setlength{\headheight}{18pt}
\pagestyle{fancyplain}
\lhead{Ken Brooks}
\chead{CS 6505 - Homework 3}
\rhead{\today}

\title{CS 6505 - Homework 3}
\author{Ken Brooks}
 
\begin{document}
\lstset{language=Python, tabsize = 4}
Answer the questions below, paying particular attention to the logic of your arguments. Short descriptions for how a function might computed (on any machine we’ve talked about) are sufficient to show that it is computable.
\begin{enumerate}
	\item Show that if A is recognizable and A reduces to $\overline{A}$ , then A is decidable.
	\\[.2in]\textbf{Answer:}
	\begin{itemize}
	\item If A reduces to B and B is recognizable, then A is recognizable. \textit{(Sipser, theorem 5.28)}
	\item In this case B is $\overline{A}$, hence $\overline{A}$ is recognizable.
	\item A language is decidable iff it is Turing-recognizable and co-Turing recognizable. \textit{(Sipser, theorem 4.28)}
	\item Hence A is decidable
	\end{itemize}
	\item Let $ B = \{\langle M \rangle \mid \mbox{M accepts exactly one of the strings 00 and 11\}}$.
		\begin{enumerate}
		\item What does Rice's theorem say about B?
			\\[.2in]\textbf{Answer:}
			\begin{itemize}
			\item The property P, that M accepts exactly one of the strings 00 and 11, is non-trivial, therefore B is undecidable.
			\end{itemize}
		\item Show that halting problem reduces to B.
			\\[.2in]\textbf{Answer:}
			\begin{itemize}
			\item Show a reduction from $H_{TM} \leq_{M}B$, where H = $\{<M> \mid \mbox{M halts on } \epsilon\}$
			\begin{lstlisting}
			def R(<M>):
				def N(x):
					if x == ``00'':
						M("")
						return Accept
					elif x == ``11'':
						return Reject
					else:
						return Reject
				return(N)
			\end{lstlisting}
			\item Commentary on the reduction:
			\begin{itemize}
				\item If N is in B then N accepts 00 or 11, but not both.  By construction N doesn't accept 11, so it must accept 00.  The only way it can accept 00 is when $M(\epsilon)$ either accepts or rejects.
				\item If N is not in B then it can't accept on the previous conditions: N must accept either both 00 and 11, or neither. By construction N rejects 11, which means it can't also reject 00. Again by construction N can't ever reject 00: it either accepts 00 or it loops.
			\end{itemize}


			\end{itemize}

		\item Show that halting problem reduces to the complement of B.
			\\[.2in]\textbf{Answer:}
			\begin{itemize}
			\item  $H_{TM} \leq_{M}\overline{B}$ via a similar reduction as above:
			\begin{lstlisting}
				def N(x):
					if x == ``00'':
						M("")
						return Accept 
					elif x == ``11'':
						return Accept
					else:
						return Reject
				return(N)
			\end{lstlisting}
			\item Commentary on the reduction:
			\begin{itemize}
				\item If N is in $\overline{B}$ then N accepts either both 00 and 11, or neither.  By construction N accepts 11, so it must also accept 00.  The only way it can accept 00 is when $M(\epsilon)$ halts (accepts or rejects). If $M(\epsilon)$ loops, then N will either loop or reject.
				\item If N is not in $\overline{B}$ then then N accepts 00 or 11, but not both. By construction N accepts 11, which means it can't also accept 00, which it does. 
			\end{itemize}

			\end{itemize}
		\item Are B or its complement recognizable?
			\\[.2in]\textbf{Answer:}
			\begin{itemize}
			\item $H_{TM}$ is recognizable, $\overline{H_{TM}}$ is not.
			\item But since $H_{TM} \leq_{M}B$, we know $\overline{H_{TM}} \leq_{M} \overline{B}$ and since $\overline{H_{TM}}$ is unrecognizable, then $\overline{B}$ is unrecognizable.
			\item And since $H_{TM} \leq_{M} \overline{B}$, we know $\overline{H_{TM}} \leq_{M} B$ and since $\overline{H_{TM}}$ is unrecognizable, then B is unrecognizable.
			\item In summary, both B and its complement are unrecognizable.
			\end{itemize}
		\end{enumerate}
\newpage
	\item A language is \textit{co-recognizable} if its complement is recognizable. Argue why each of the following languages is or is not recognizable and why it is or is not co-recognizable.
		\begin{enumerate}
		\item $L1 = \{\langle M \rangle \mid \mbox{M enters state q27 for some input string x}\}$.\footnote{NB: This is the dead code problem, which is not decidable}
			\\[.2in]\textbf{Answer:}
			\begin{itemize}
			\item We can reduce $H_{TM}$ to L1 by defining a new machine M' that halts if M enters q27.
			\item $H_{TM}$ is not decidable, hence L1 is not decidable.
			\item $H_{TM}$ is recognizable (you can run it until it halts), but $\overline{H_{TM}}$ is not.
			\item $\overline{H_{TM}}$ reduces to $\overline{L1}$. \textbf{Proof reference}
			\item $\overline{H_{TM}}$ is not recognizable, therefore $\overline{L1}$ is not recognizable.
			\item Consequently L1 is recognizable, but not co-recognizable.
			\end{itemize}
		\item $L2 = \{\langle M \rangle \mid \mbox{L(M) contains at most two strings}\}$.
		\\[.2in]\textbf{Answer:}
			\begin{itemize}
			\item Take the complement of L2: $\overline{L2} = \{\langle M \rangle \mid \mbox{L(M) contains at least two strings}\}$.
			\item $A_{TM}$ is recognizable.
			\item Reduce $A_{TM}$ to $\overline{L2}$ by running $A_{TM}$ on successive w's via dovetailing until you've found at least 2 strings.
			\item This demonstrates that $\overline{L2}$ is recognizable.
			\item Via Rice's theorem we know that L2 is undecidable, and via Sipser, theorem 4.28 
			\item Via Sipser, theorem 4.28 we know that if L2 is undecidable and $\overline{L2}$ is recognizable, that L2 must be unrecognizable.
			\item Consequently L2 is unrecognizable, but is co-recognizable.
			\end{itemize}
		\item $L3 = \{\langle M \rangle \mid L(M) \subseteq \Sigma^{*}\}$
		\\[.2in]\textbf{Answer:}
			\begin{itemize}
			\item This language accepts everything in the powerset of the alphabet which is the universal language that accepts everything, $L_{U}$.
			\item From Lance's Lesson from last week $L_{U}$ is recognizable. 
			\item Since from Rice's theorem we know that $L_{U}$ is undecidable, $\overline{L_{U}}$ is not recognizable.
			\item Consequently L3 is recognizable, but is not co-recognizable.
			\end{itemize}
	\end{enumerate}
\newpage
	\item A computable verifier is a deterministic Turing machine $V$ that takes two arguments: $x$ (the input) and $y$ (the proof). A computable verifier always halts. Show that a language $L$ is recognizable if and only if there exists a computable verifier $V$ such that
		\begin{enumerate}
		\item if $x \in L$ , then there is a string $y$ such that $V(x,y)$ accepts, and
		\item if $x \notin L$, then $V(x,y)$ rejects for every string $y$.
	\\[.2in]\textbf{Answer:}
	\begin{itemize}
	\item \textit{Theorem: }If a language $L$ is recognizable then there exists a computable verifier $V$.
	\begin{itemize}
	\item If L is recognizable then there is a machine, M, that accepts every string in L.
	\item Let $y = \epsilon$ and concatenate $y$ to every string in L.
	\item M is a computable verifier for L since it recognizes every string in L in the presence of $y$.
	\end{itemize}
	\item \textit{Theorem: } If there exists a computable verifier $V$, then language $L$ is recognizable.
	\begin{itemize}
	\item A computable verifier $V(x,y)$ is a decider for L in the presence of y: By definition it accepts when $x$ is in $L$ and rejects when $x$ is not in $L$.
	\item If a string $x$ is in L then $V(x,y)$ will accept it.
	\item L is recognizable.
	\end{itemize}
	\item Hence a language L is recognizable if and only if there exists a computable verifier V such that
		\begin{enumerate}
		\item [(a)]if $x \in L$ , then there is a string y such that V(x,y) accepts, and
		\item [(b)]if $x \notin L$, then V(x,y) rejects for every string y.
		\end{enumerate}
	\end{itemize}
\end{enumerate}
\item Consider the following property: $P = \{\langle M \rangle \mid$ L(M) is accepted by some Turing machine that has an odd number of states\}. Show that P is a trivial property.
		\\[.2in]\textbf{Answer:}
	\begin{itemize}
	\item Modify the Turing machine in question by adding an inactive state.  
	\item L(M) is now accepted by some Turing machine with an even number of states. 
	\item The same language L(M) is now accepted regardless of whether the property P applies
	\item Consequently the property is trivial.
	\end{itemize}
\newpage
\item Bonus: Read about The Recursion Theorem in the Sipser text. One implication of the recursion theorem is that in any general purpose programming language, one can write code that outputs the code itself. Write a python program that prints its own code. Do not use any file operations.
\\[0.2in] \url{https://www.udacity.com/course/viewer#!/c-ud557/l-1209378918/m-2986218594}
\\[.2in]\textbf{Answer:}
\\[0.2in] I submitted the following code to the link above.  The answer was accepted.
		\begin{lstlisting}
		x='x=%s\nprint x%%`x`'
		print x%`x`
		\end{lstlisting}

\end{enumerate}
\end{document}